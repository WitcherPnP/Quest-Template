\mychapter{0}{Vorwort}
Dieses Quest-Dokument darf nur vom SL gelesen werden. Generell gilt, dass dieses Dokument keinen roten Faden beinhaltet, dem die Spieler bzw. der SL folgt. Stattdessen wird die Grundidee der Quest beschrieben, in dem mehrere unabhängige Ereignisse beschrieben werden, die zwar keine feste Reihenfolge festlegen müssen, dem SL aber helfen den Spielern den richtigen Weg zu weisen. Somit liegt der Verlauf und Ausgang des Abenteuers allein in der Hand des Spielleiters (SL). Das Dokument ist im Wesentlichen in drei Teile aufgeteilt, um den SL bei der Erzählung der Quest zu unterstützen.

Das erste Kapitel beschreibt die Hauptquest, die die Spieler beschreiten sollen. Darin werden einzelne Passagen bzw. Ereignisse beschrieben, die unabhängig voneinander sind und den Spielern vom SL\footnote{In diesem Dokument werden Abkürzungen aus dem Abkürzungsverzeichnis des Grundregelwerks benutzt.} vorgelesen werden können. Wann diese Ereignisse eintreffen, kann der SL entscheiden. Dazu kann er NPCs so agieren lassen, dass es zu einem der beschriebenen Ereignissen kommt um so der Verlauf der Gischichte voranzutreiben. Wenn es beispielsweise ein Ereignis gibt, das beschreibt wie die Spieler verhaftet und in den Kerker gebracht werden, kann der SL die Spieler durch einen NPC dazu anstiften etwas illegales zu tun, damit es zur Verhaftung kommt. Das kann aber auch bedeuten, dass nicht alle Ereignisse im Laufe des Abenteuers eintreten. Sowieso sind alle beschriebenen Ereignisse nur als Vorschlag zu verstehen und können jederzeit beliebig vom SL angepasst werden. Zu jedem Ereignis gehören auch zusätzliche Informationen wie mögliche Aktionen der Spieler. D.h., dass genau definiert ist, was sie z.B. sehen wenn sie tagsüber in der Herberge aus dem Fenster schauen, oder welche Probe sie mit welcher QS schaffen müssen um eine versteckte Truhe zu entdecken.

Das zweite Kapitel beschreibt alle Orte, die die Spieler voraussichtlich innerhalb der Quest besuchen werden oder können. Da die Spieler theoretisch überall hingehen können, ist es kaum möglich alle Orte zu beschreiben, aber zumindest die wichtigsten. Bei den Orten kann es sich um Städte, Gebäude (Rathaus, Herberge), Stadtteile (Hafen, Marktplatz), Sehenwürdigkeiten und vieles mehr handeln. Hauptsache ist, dass es für die Spieler, nicht unbedingt für die Quest, relevant ist. D.h., dass auch für die Quest irrelevante Orte wir Herbergen oder diverse Geschäfte aufgeführt werden können. Diese Orte sind dann für die Spieler von Interesse, weil sie dort z.B. neue Ausrüstungsgegenstände erwerden können. Ein Ort besteht (optional) aus einem Bild/einer Zeichnung und einer Textpassage für die Spieler, die den Ort beschreibt. Handgemalene Zeichnungen können auch nachträglich zum Dokument hinzugefügt werden. Zusätzlich können wie im ersten Kapitel auch, Proben und mögliche Aktionen der Spieler festgelegt sein. Sobald die Helden zu einem der beschriebenen Orte gelangt, kann sich der SL das entsprechende Kapitel zur Hilfe nehmen. Der SL kann außerdem versuchen durch geschicktes agieren der NPCs, die Spieler zu Quest-relevanten Orten zu führen. 

Damit der SL die NPCs glaubwürdig agieren lassen kann, sind im letzten Kapitel alle wichtigen NPCs aufgeführt. Diese kann der SL übernehmen, um mit den Spielern zu interagieren und ihnen so den richtigen (oder auch falschen) Weg zu zeigen. Jeder NPC in diesem Dokument enthält allgemeine Informationen zum Aussehen, eine Hintergrundgeschichte und Motive, die für die Quest und Spieler relevant sein können. Außerdem können sie genauso wie die Spieler bestimmte Werte für den Kampf oder andere Talente haben.