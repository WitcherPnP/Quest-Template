\chapter{Questablauf}
Hier wird die eigentliche Idee der Quest beschrieben. Es ist wichtig, zu welcher Zeit (Jahr und evtl. Jahreszeit) die Quest spielt, den Spieler die Ereignisse zu schildern, mögliche Optionen aufzeigen (nur für den SL einsehbar), welche Proben für bestimmte Sachen abgelegt werden müssen, wie die Spieler an wichtige Informationen und Quest-Gegenstände kommen bzw. wie diese aussehen, aber auch beispielsweise Nebenquests und mögliche Belohnungen.

\section{Hintergrundinformationen}
\textit{(Zu welcher Zeit spielt die Quest, und was passiert gerade in der Welt? Optional kann dieser Teil den Spielern vorgelesen werden.)}

\textbf{21. Juli 1264} - Vor ca. einem Monat ist die Nilfgaarder Armee im Süden von Lyrien einmarschiert. Vor 17 Tagen haben sie zuerst die Festungen Scala und Spalla und nur wenige Tage später die Stadt Riva erobert. Demawends Armee ist im Pontar-Tal stationiert und wartet auf Befehle des Königs, während die Truppen von Mewe, Königin von Rivien und Lyrien, langsam aber stetig bis in den Norden zurückgedrängt werden. 

Zur gleichen Zeit greifen immer mehr in den Mahakam-Bergen lebende Scoia'tael-Banden Dörfer und schwach bewachte Lager an, um die Nördlichen Armeen zu flankieren. Sie überfallen und blockieren wichtige Handelsrouten, die für die Versorgung der nördlichen Armeen essentiell sind.

Da König Wisimir vor wenigen Monaten ermordet wurde, herrscht noch immer Chaos in Redanien. Da der Sohn von Wisimir erst 13 Jahre alt ist, übernahm der Thronrat von Redanien die Macht. Man wollte das Königreich auf gar keinen Fall in die Hände von Wisimirs Frau, Königin Hedwig legen. Um dem Chaos im eigenem Lande Herr zu werden stationierte der Thronrat, trotz Hedwigs Proteste, 1000 Soldaten der königlichen Armee an der temerischen und redanischen Ostgrenze zu Aedirn bzw. Kaedwen, da man befürchtete, die Scoia'tael oder schlimmer die Schwarzen könnten bis nach Temerien, Kaedwen und Redanien vorstoßen. 

An dieser Stelle knüpft das epische Abenteuer der Spieler an.


\section{Kapitel 1 - Das Abenteuer beginnt}
\textit{(Einstieg für die Spieler ins Abenteuer.)}

\textit{Die Helden wachen noch leicht angetrunken in einem großen Raum auf. Jeder liegt auf einem Bett. 8 weitere Betten stehen im Raum. Alle sind leer. Durch die kaputten Fensterläden scheint die Sonne und erhellt das Zimmer. Es gibt eine Tür und auf zwei Seiten Fenster.} 

\textit{(Mögliche Optionen)}

\textbf{Fenster}: Sie sind im zweiten Stock. Blick auf eine mittelmäßig belebte Straße.\\
\textbf{Tür}: Nicht verschlossen. Führt in einen Flur. (siehe Zeichnung)\\
\textbf{Bett(en)}: Unter den eigenen Betten der Spieler sind ihre Sachen (Rucksäcke).\\
\textbf{Mögliche Schätze im Gebäude}: Lebensmittel, Schmuck (von Gästen).\\

\textit{(Informationen, die die Spieler herausfinden können)}

\textbf{Informationen}: Gasthaus, "`Goldener Drache"'. Gestern gab es eine große Feier mit Livemusik und viel Alkohol, um die aktuelle Situation für einen Abend zu vergessen. Außerdem haben die Spieler den Geburtstag von Annika gefeiert. (reingefeiert) \\

\textit{(Mögliche Talentproben. Beispiel: Wenn jemand eine Probe auf Sinnesschärfe mit QS I oder höher schafft, entdeckt er eine (leicht) versteckte Truhe.)}

\textbf{Sinnerschärfe I}: An einer Wand ohne Fenster steht zwischen zwei Betten eine Truhe.\\

\textit{(Optionale? Nebenquests sind gleich aufgebaut, wie die eigentlich Quest.)}

\textbf{Nebenquest} - ...

\textit{(Wichtige (Hintergrund-)Informationen für den SL)}

\textbf{Wichtige Information (zu einem späteren Zeitpunkt)}:

\textit{(Weiterer Verlauf der Hauptquest)}

\textbf{Hauptquest} - ...
