{\let\clearpage\relax\chapter{Orte}}

\section{Hagge}
\textit{(Beschreibung des Ortes. Wird den Spielern vorgelesen.)}

Vor Hagge - \textit{Hagge ist eine relativ große Stadt, die direkt am Pontar liegt. Sie ist umgeben von großen robusten Steinmauern. Sowohl das Ost-, als auch das Westtor werden von Soldaten in Rot bewacht. Um Hagge herum sind offensichtlich mehrere Soldaten auf Patroullie.}\\ 

\textit{(Mögliche Proben, damit die Spieler an Infos kommen.)}

\textbf{Menschenkenntnis I+} - Sie wirken sehr unruhig und besorgt. \\
\textbf{Etikette I} - Es scheint sich um redanische Soldaten zu handeln. \\
\textbf{Etikette II} - Anhand des Abzeichens kann man erkennen, dass es sich um irgendeine spezielle Division handelt.\\
\textbf{Etikette III+} - Eine redanische Spezial-Einheit die als kampfstarke Spähereinheit eingesetzt wird. Sie ist mit einigen Rennpferden ausgestattet, damit sie schnell Nachrichten überbringen können. 

\textit{Weitere Infos für die Spieler zum Ort.}

In Hagge (tagsüber) - \textit{Auch in Hagge sind viele Soldaten auf Patroullie. Die Seitenstraßen wirken wie leergefegt. Nur auf der Hauptstraße, die vom West- zum Osttor führt, und am Hafen sammeln sich viele Leute. Dort sind viele Zelte aufgestellt worden, die scheinbar für arme Leute} (Flüchtlinge) \textit{als unterkunft dienen.}

\textit{(Optional: Bild oder Zeichnung vom Ort.)}

\textit{Mögliche Orte innerhalb, die die Spieler besuchen können. Mögliche Referenzen auf das obige Bild bzw. die obige Zeichnung.}

\subsection{Marktplatz} 
\textit{(Optional: Bild oder Zeichnung vom Ort.)}

\textit{(Infos wie Name, Ort und eine Beschreibung für die Spieler.)}

Stadtkern\\
Wenige bis gar keine Händler. Zum Flüchtlingslager umfunktioniert. \\

\textit{Am Hafen ist es noch voller als sonst. Zwischen den ganzen} (Flüchtlings-)\textit{Zelten wirren zusätzlich Hafenarbeiter umher, die die wenigen Schiffe mit Kisten entladen und beladen. Arbeiter schleppen Kisten und Nahrung in die Lager. Es gibt niemanden der nichts zu tun hat.}

\textit{(Mögliche Proben, damit die Spieler an Infos kommen.)}

\textbf{Etikette I} - Die Flüchtlinge scheinen der Kleidung her einfache Bauern zu sein.\\
\textbf{Etikette II} - Es gibt auch wenige sozial Höhergestellte unter den Flüchtlingen, die scheinbar aus dem Süden kommen. Sie haben eigene Zelte oder zumindest mehr Platz in den Gemeinschaftszelten.
